\documentclass[12pt]{article}

%Packages
\usepackage[spanish,es-lcroman,es-nosectiondot]{babel}
\usepackage[utf8]{inputenc}
\usepackage{amsmath}
\usepackage{mathrsfs}
\usepackage{amsfonts}
\usepackage{comment}
\usepackage{amsthm}

\usepackage{mathtools}
\usepackage{amssymb}

\usepackage{systeme}

%Graphics
\usepackage{graphicx}
%\usepackage{hyperref}
\graphicspath{{./}}
%\usepackage{mathtools}
\usepackage{adjustbox}

\usepackage{fancyhdr}
\usepackage{lipsum}

%Auxiliar package 
\usepackage{float} 
%\usepackage[normalem]{ulem}
%\useunder{\uline}{\ul}{}
\usepackage{textcomp}
%Document margins
\usepackage[margin=2cm]{geometry}

%Title definition
\author{}
\title{Informe: Sentido Animal}
\date{}

%Alias definitions
\let\bs\textbackslash

%Variables definitions
\newcommand{\nls}{\\[0.10pt]}
\newcommand{\ngls}{\\[0.20cm]}
\newcommand{\nlsm}{\\[0.3cm]}
\newcommand{\nin}{\noindent}

\newcommand{\dps}{$: \ $}
\newcommand{\limit}{\lim_{n \to \infty}}

%More definitions
\newtheorem{law}{Ley}

\theoremstyle{definition}
\newtheorem*{defn}{Definición}
\newtheorem*{note}{Nota}
\newtheorem*{theo-ne}{}
%Delete indentation
%\setlength{\parindent}{0pt}

\def\N{\mathbb{ N}}
\def\R{\mathbb{ R}}
\def\S{\mathbb{ S}}
\def\O{\mathnormal{ O}}
\def\Om{\Omega}

%\pagestyle{fancy}
%\fancyfoot{}{}{}
%\fancyhead[LO,RE]{\leftmark}
%\fancyhead[LE,RO]{\thepage}
     
 

\pagestyle{fancy}
%\renewcommand{\sectionmark}[1]{ \markright{#1}{} }
\fancyhf{}
%\rhead{\rightmark}
\fancyhead[L,L]{Tecnologías Web}
\fancyhead[R,R]{Informe Subproblema-2}
%\fancyfoot[L,L]{\thepage}
%\lhead{\scshape \nouppercase{\firstmark}}
\cfoot{\thepage}
\renewcommand{\footrulewidth}{0.4pt}
%\renewcommand{\thesection}{\arabic{section}}
%\renewcommand\sectionmark[1]{%
%   \markright{\thesection\ #1}}

\usepackage{invoice}
%\usepackage{longtable}

\renewcommand{\Activity}{Descripción}
\renewcommand{\SumFees}{Subtotal}
\renewcommand{\SubtotalProject}{Subtotal + IVA}
\renewcommand{\SubtotalFee}{Subtotal acumulado}

\setlength{\parindent}{0pt}

\begin{document}

\maketitle

\section*{Información básica}

Sentido Animal es una asociación sin ánimo de lucro, centrada en la protección de los animales. Los datos relevantes de esta entidad son:

\begin{itemize}
\item[•] Identidad del Responsable: Charo Peromingo Jiménez
\item[•] Domicilio fiscal: Calle Aristóteles Nº 176, 04700, El Ejido (Almería)
\item[•] CIF: G04889614
\item[•] Correo de contacto: sentidoanimal.elejido@gmail.com
\end{itemize}

Dicha asociación dedica fundamentalmente su esfuerzo hacia:

\begin{itemize}
\item[•] Recoger perros y gatos abandonados en la calle.
\item[•] Fomentar la adopción de perros 
\item[•] Aplicar la metodología CER a los gatos (Castración, Esterilización y Reubicación).
\end{itemize}

\section*{Estado actual}

La asociación descrita carece actualmente de página web. Su principal fuente de difusión y promoción son las redes sociales, fundamentalmente \textit{Instagram} y \textit{Facebook}.\\

Carecen de sede física fija donde realizar las actividades descritas. Para ello se nutren de distintas casas de acogida que permiten la distribución de los animales rescatados.

\section*{Requisitos y funcionalidades}

La asociación solicita los siguientes requisitos:

\begin{itemize}
\item[•] Que se disponga de una página principal que contenga la información relevante, así como imágenes destacadas.
\item[•] Que los usuarios de la página puedan adscribirse a la misma. Dichos usuarios adscritos podrán realizar donaciones, comprar en la tienda online, solicitar ser casa de acogida y modificar su perfil si fuera necesario.
\item[•] Que la asociación pueda modificar cierto contenido de la página: imágenes, textos, etc, en virtud de sus necesidades.
\end{itemize}


En resumen, la asociación solicita un gestor de contenidos que ellos mismos puedan gestionar desligándose del código asociado cuyo mantenimiento queda al cargo de EL EQUIPO DESARROLLADOR.\\

Para ello se acuerdan las siguientes funcionalidades:

\subsection*{Contenidos de la página web}

Se distingue fundamentalmente entre información estática (aquella asociada al código) e información dinámica (aquella susceptible a cambios).\\

\subsubsection*{Contenido estático}

El principal contenido estático será la infraestructura del propio gestor de contenidos. 

\subsubsection*{Contenido dinámico}

El contenido dinámico será: la dotación para gestionar usuarios, productos y formularios, las imágenes de las páginas, así como el contenido multimedia asociado. 

Dicho listado podrá sufrir modificaciones durante el proceso de desarrollo.


\subsection*{Base de datos}

La base de datos estará creada con MySQL y se gestionará con PHP. Dicha base de datos deberá contener:

\begin{itemize}
\item[•] Información sobre los usuarios y todos los derivados propios de la administración.
\item[•] Información sobre los productos alojados en la web.
\item[•] Información sobre el contenido multimedia y su almacenamiento.
\end{itemize}

El contenido multimedia podría adscribirse a servidores CDN.


\subsection*{Perfiles de usuario}

Se definen los siguientes perfiles de usuario:

\begin{itemize}
\item[•] Perfil de cliente: Se trata del usuario más básico cuyo rol es el de consumidor.
\item[•] Perfil de administrador: Se trata de una extensión del perfil de cliente que puede realizar modificaciones en la página web por medio del gestor de contenidos.
\item[•] Perfil de súper-administrador: Se trata de una extensión del perfil de administrador que puede realizar cambios en el propio gestor de contenidos. En este caso, este rol recae sobre EL EQUIPO DESARROLLADOR.
\end{itemize}

\subsection*{Menú con las principales funciones}

Los menús variarán en función del usuario:

\begin{itemize}

\item[•] Perfil de cliente: las principales funciones serán: \textit{realizar donación}, \textit{realizar compra}, \textit{solicitar ser casa de acogida } y \textit{modificar perfil}.

\item[•] Perfil de administrador: se extienden las funcionalidades a: \textit{gestionar productos}, \textit{gestionar contenido multimedia}, \textit{gestionar formularios de casa de acogida}.

\item[•] Perfil de súper-administrador: tiene todas las funcionalidades posibles, añadiendo: \textit{gestionar gestor de contenidos} que permitirá realizar modificaciones en el gestor de contenidos. Por otro lado, las funcionalidades de este usuario se realizarán directamente sobre los ficheros en el servidor.

\end{itemize}

Estas funcionalidades deberán extenderse y podrían sufrir modificaciones en futuras resoluciones.

\subsection*{Plan de desarrollo y evaluación}

El plan de desarrollo se seguirá usando las herramientas y criterios descritos en el documento \textit{Guía de Buenas Prácticas} adjunto en el Subproblema-1.

\end{document}



